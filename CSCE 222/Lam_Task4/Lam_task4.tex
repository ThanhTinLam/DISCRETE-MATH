\documentclass{article}
\usepackage{amsmath}
\usepackage{amssymb}
\usepackage{amsfonts}
\newtheorem{theorem}{Theorem}[section]

\title{\textbf{MATHEMATICAL LOGIC}}
\author{ Thanh Tin Lam}
\date{ 05 October 2023}

\begin{document}
\maketitle
\vspace{2cm}
\section{Principle of Mathematical Induction}

\begin{theorem}{\textbf{The five Peano axioms:}}
    \begin{itemize}
        \item Zero is a natural number.
        \item Every natural number has a successor in the natural numbers.
        \item Zero is not the successor of any natural number.
        \item If the successor of two natural numbers is the same, then the two original numbers are the same.
        \item If a set contains zero and the successor of every number is in the set, then the set contains the natural numbers.
    \end{itemize}
\end{theorem}

\hspace{2mm} The fifth axiom is known as the principle of induction because it can be used to establish properties for an infinite number of cases without having to give an infinite number of proofs. In particular, given that P is a property and zero has P and that whenever a natural number has P its successor also has P, it follows that all natural numbers have P.\\
\vspace{1mm}

\hspace{2mm} Mathematical induction is a powerful proof technique that can be applied in many different areas of mathematics and computer science, can refer to combinatorics, number theory, discrete mathematics or proving inequalities. In the field of computer science, induction is used in recursive data structures and algorithms to prove correctness.

\end{document}
